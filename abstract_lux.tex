\documentclass[12pt]{article}

\usepackage[T1]{fontenc}
\usepackage[utf8]{inputenc}
\usepackage[english]{babel}
\usepackage{fullpage}

\title{Accountability and Privacy in Gossip}
\author{Jérémie Decouchant}
\date{September 18, 2015}

\begin{document}
\maketitle

\textbf{Abstract:} Gossip-based content dissemination protocols are a scalable and cheap 
alternative to centralized content sharing systems. However, it is well known 
that these protocols suffer from selfish nodes, i.e., nodes that aim at 
downloading the content without contributing their fair share to the system. 
While the problem of selfish nodes that act individually has been well 
addressed in the literature, \textit{collusions} of selfish nodes are still possible. In addition, previous rational-resilient gossip-based protocols require nodes to log their interactions with others, and disclose the content of their logs, which may disclose sensitive information. Nowadays, a consensus exists on the necessity of reinforcing the control of users on their personal information. Nonetheless, to the best of our knowledge no privacy-preserving rational-resilient gossip-based content dissemination system exists.\\
In this presentation, we will discuss two recent contributions. First, we will present the mechanisms of AcTinG, the first gossip protocol that prevents collusions of selfish nodes. Second, we will describe $PAG$, which is a gossip protocol that prevents selfish behaviors while preserving the privacy of its users. \\

\textbf{Jérémie Decouchant} is completing his PhD at the University of Grenoble, France. His research took place in the ERODS team (Efficient and RObust Distributed Systems), which is part of the LIG (Laboratory of Informatics at Grenoble) at Grenoble, France. His research interests include distributed systems, privacy and accountability mechanisms, and multicore systems. 

\end{document}
